\section{An Auction Based Approach}

Recall that the LP-Relaxation of the \minbfactor{} problem in a bipartite graph $G = \lrp{L \cup R, E}$  is given by
\begin{mini}
    {}{\sum_{(i, j) \in E} \f[w]{i, j} \f[x]{i, j}}{}{\label{lp:primal_verbose}}{}
    \addConstraint{\sum_{j \in N(i)} \f[x]{i, j}}{= \f[b]{i}}{\quad \forall i \in L}
    \addConstraint{\sum_{i \in N(j)} \f[x]{i, j}}{= \f[b]{j}}{\quad \forall j \in R}
    \addConstraint{0 \leq \f[x]{i, j}}{\leq 1}{\quad \forall \lrp{i,j} \in E}
\end{mini}
and its dual by
\begin{maxi}
    {}{\sum_{i \in L} \f[b]{i} \f[r]{i} + \sum_{j \in R} \f[b]{j} \f[p]{j} - \sum_{(i,j) \in E} \f[z]{i, j}}{}{\label{lp:dual_verbose}}{}
    \addConstraint{\f[r]{i} + \f[p]{j} - \f[z]{i, j}}{\leq \f[w]{i, j}}{\quad \forall \lrp{i, j} \in E}
    \addConstraint{\f[z]{i, j}}{ \geq 0}{\quad \forall \lrp{i,j} \in E}. 
\end{maxi}
For exposition purposes, we refer to vertices in $L$ as \emph{bidders}, and vertices in $R$ as \emph{objects}. 

By complementary slackness conditions (\cref{lem:comp_slack}), we find that we do not have to maintain the edge duals $z$
as their optimal value can be given explicitly by $\f[z]{i, j} = \max \lrc{\f[r]{i} + \f[p]{j} - \f[w]{i, j}, 0}$.
This further implies that for all $\lrp{i, j} \in E$, $\f[r]{i} + \f[p]{j} \leq \f[w]{i, j}$. To condense notation, we define $\f[w^\prime]{i, j} = \f[w]{i, j} - \f[p]{j}$ as the \emph{augmented cost} 
of object $j$ for bidder $i$. This gives us that the dual can be equivalently described by 
\begin{maxi}
    {}{\sum_{i \in L} \f[b]{i} \f[r]{i} + \sum_{j \in R} \f[b]{j} \f[p]{j} - \sum_{(i,j) \in E} \max \lrc{\f[r]{i} - \f[w^\prime]{i, j}, 0}}{}{\label{lp:dual_verbose_alt}}{}
    \addConstraint{\f[r]{i} + \f[p]{j}}{\leq \f[w]{i, j}}{\quad \forall \lrp{i, j} \in E}.
\end{maxi}

If we are given a price function $p$, then the objective value of \cref{lp:dual_verbose_alt} is maximized when $\f[r]{i} = \min_{k \in N(i)} \f[w^\prime]{i, k}$ while still
respecting the constraints of the problem. Since the edge duals $z$ depend on $r$, they can be given explicitly by $\f[z]{i, j} = \max \lrc{\min_{k \in N(i)}\f[w^\prime]{i, k} - \f[w^\prime]{i, j}, 0}$. 
However, since clearly $\min_{k \in N(i)}\f[w^\prime]{i, k} \leq \f[w^\prime]{i, j}$, all the edge duals are given by $\f[z]{i, j} = 0$. 
Hence, the dual problem can be further equivalently described by 
\begin{maxi}
    {p}{\sum_{j \in R} \f[b]{j} \f[p]{j} + \sum_{i \in L} \f[b]{i} \min_{k \in N(i)} \f[w^\prime]{i, k}}{}{\label{lp:dual_price}}{}
\end{maxi}

\subsection{$\varepsilon$-Happiness}

\begin{comment}
By complementary slackness conditions in \cref{lp:dual_price}, if $\f[x]{i, j} > 0$ in an optimal primal solution, then 
\begin{equation}
    \f[w]{i, j} - \f[p^{\ast}]{j} = \f[r^{\ast}]{i}
    \label{eq:comp-slack-alt}
\end{equation}
where $p^{\ast}$ and $r^{\ast}$ are optimal dual solutions. If we are given the optimal price 
function, however, then \cref{eq:comp-slack-alt} is equivalent to
\begin{equation}
    \f[w^\prime]{i, j} = \min_{k \in N(i)} \f[w^\prime]{i, k}
    \label{eq:happy-edge}
\end{equation} 
as $r^{\ast}$ is given as in the previous section. 
\end{comment}

Let $F^{\ast}$ be an optimal $b$-factor and $p^{\ast}$ an optimal price function. Consider the incidence function $x$ induced by $F^{\ast}$. 
By complementary slackness conditions in \cref{lp:dual_price}, this implies that if $\f[x]{i, j} > 0$,
\begin{equation}
    \begin{aligned}
        \f[w]{i, j} - \f[p^{\ast}]{j} &= \min_{k \in N(i)} \f[w^\prime]{i, k}\\
        \f[w^\prime]{i, j} &= \min_{k \in N(i)} \f[w^\prime]{i, k}. 
    \end{aligned}
    \label{eq:happy-edge} 
\end{equation}
\begin{comment}
Since these are primal and dual optimal, respectively, by strong duality
\begin{align*}
    \sum_{(i, j) \in F^{\ast}} \f[w]{i, j} &= \sum_{j \in R}  \f[b]{j} \f[p^{\ast}]{j} +  \sum_{i \in L}  \f[b]{i} \min_{k \in N(i)} \f[w^\prime]{i, k}\\
    \sum_{(i, j) \in F^{\ast}} \f[w]{i, j} - \f[p^{\ast}]{j} &= \sum_{i \in L}  \f[b]{i} \min_{k \in N(i)} \f[w^\prime]{i, k}\\
    \sum_{(i, j) \in F^{\ast}} \f[w^\prime]{i, j} &= \sum_{i \in L}  \f[b]{i} \min_{k \in N(i)} \f[w^\prime]{i, k}
\end{align*}
which implies that for all $\lrp{i, j} \in F^{\ast}$,
\begin{equation}
    \f[w^\prime]{i, j} =  \min_{k \in N(i)} \f[w^\prime]{i, k}. 
    \label{eq:happy-edge}
\end{equation}
\end{comment}
For any feasible simple $b$-matching (not necessarily a $b$-factor) $F$ and price function $p$, we call an edge in $F$ that satisfies \cref{eq:happy-edge} \emph{happy} with respect to $p$. 
We can relax this condition to consider bidders assigned to objects within $\varepsilon$ of attaining the minimum augmented cost, where $\varepsilon \geq 0$. That is,
\begin{equation}
    \f[w^\prime]{i, j} \leq \min_{k \in N(i)} \f[w^\prime]{i, k} + \varepsilon. 
    \label{eq:epsilon-happy-edge}
\end{equation}
If an edge in $F$ satisfies \cref{eq:epsilon-happy-edge}, then it is called \emph{$\varepsilon$-happy} with respect to $p$. 
If $F$ consists of only $\varepsilon$-happy (resp.\! happy) edges, then it is also $\varepsilon$-happy (resp.\! happy) with respect to $p$. If $F$ is a $b$-factor and is happy, then it must be optimal by \cref{eq:happy-edge}. 
However, if it is a $b$-factor and $\varepsilon$-happy, then its weight falls within an additive loss of the optimal min weight $b$-factor.  
%Similarly, a $b$-factor $F$ that consists of only $\varepsilon$-happy edges is called $\varepsilon$-happy. 
%If a $b$-factor is $\varepsilon$-happy, then its weight falls within an additive loss of the optimal min weight $b$-factor. 

\begin{lemma}
    Let $F, p$ be some feasible $b$-factor and price function, respectively, such that $F$ is $\varepsilon$-happy with respect to $p$. Then 
    \begin{equation*}
        \f[w]{F^{\ast}} \leq \f[w]{F} \leq \f[w]{F^{\ast}} + \varepsilon U
    \end{equation*}
    where $F^{\ast}$ is an optimal $b$-factor and $U = \frac{1}{2} \f[b]{V} = \abs{F}$. 
    \label{lem:additve_loss}
\end{lemma}
\begin{proof}
    Since $F$ is $\varepsilon$-happy, each edge $\lrp{i, j} \in F$ satisfies \cref{eq:epsilon-happy-edge}, which implies that 
    \begin{equation*}
        \min_{k \in N(i)} \f[w^\prime]{i, k} \geq \f[w]{i, j} - \f[p]{j} - \varepsilon. 
    \end{equation*}
    Additionally, by definition of a $b$-factor, 
    \begin{equation*}
        \sum_{j \in R}  \f[b]{j} \f[p]{j} +  \sum_{i \in L}  \f[b]{i} \min_{k \in N(i)} \f[w^\prime]{i, k} = \sum_{(i, j) \in F} \lrp{\f[p]{j} + \min_{k \in N(i)} \f[w^\prime]{i, k}}
    \end{equation*}
    since each vertex $v \in L \cup R$ is matched exactly $\f[b]{v}$ times in $F$. 
    Hence, by weak duality, 
    \begin{align*}
        \f[w]{F^{\ast}} &\geq \sum_{j \in R}  \f[b]{j} \f[p]{j} +  \sum_{i \in L}  \f[b]{i} \min_{k \in N(i)} \f[w^\prime]{i, k}\\
        &= \sum_{(i, j) \in F} \lrp{\f[p]{j} + \min_{k \in N(i)} \f[w^\prime]{i, k}} \\
        \f[w]{F^{\ast}} &\geq  \sum_{(i, j) \in F} \lrp{\f[w]{i, j} - \varepsilon} = \f[w]{F} - \varepsilon U \geq \f[w]{F^{\ast}} - \varepsilon U. 
    \end{align*}
    Adding $\varepsilon U$ to the last line gives $\f[w]{F^{\ast}} + \varepsilon U \geq \f[w]{F} \geq \f[w]{F^{\ast}}$.
\end{proof}

\subsection{The Auction Algorithm}

The auction algorithm uses the alternative expression of the dual problem \cref{lp:dual_price} to find a $b$-factor by iteratively constructing a 
feasible price function $p$. In each iteration, an unsaturated bidder $u$ finds the objects $v_1, v_2$ that give it the lowest and second-lowest augmented cost, respectively, 
with the intention of bidding on the lowest cost object. The bid is calculated by simply taking the difference in the augmented costs of the two objects, plus some additional $\varepsilon$ value.
The price of object $v_1$ is then lowered by the bid, and $u$ is matched to $v_1$. If $v_1$ is already saturated in $F$ before adding the edge $\lrp{u, v_1}$, then a random 
edge in $F$ containing $v_1$ is removed and replaced with $\lrp{u, v_1}$. This process continues as long as there are unsaturated bidders, and if it 
terminates, it must necessarily return a feasible $b$-factor. 

\begin{algorithm}[!h]
    \caption{Sequential Auction} \label{alg:auction}
    \begin{algorithmic}[1]
        \AlgIn A bipartite graph $G = \lrp{V = L \cup R, E}$, edge weight function $w \colon E \to \R_{\geq 0}$, node capacity function $b \colon V \to \Z_+$, $\varepsilon \geq 0$, and prices $p\colon V \to \R$
        \AlgOut A $b$-factor $F$ and price function $p$ such that $F$ is $\varepsilon$-happy with respect to $p$

        \State $U \gets L$ \Comment{Unsaturated nodes}
        \State $F \gets \emptyset$
        \While{$U \neq \emptyset$}
            \State Choose any $u \in U$
            \State $\Call{Bid}{u, p, U, F}$
        \EndWhile
        \State \Return $\lrp{F, p}$
        \State
        \Procedure{Bid}{$u$, $p$, $U$, $F$}
            \State $v_1 \gets \argmin_{v \in N(u)} \f[w^\prime]{u, v}$ \Comment{Lowest augmented cost object}
            \State $v_2 \gets \argmin_{v \in N(u) \sm \lrc{v_1} } \f[w^\prime]{u, v}$ \Comment{Second lowest augmented cost object}
            \State $\gamma \gets \f[w^\prime]{u, v_2} - \f[w^\prime]{u, v_1} + \varepsilon$ \Comment{Bid}
            \State $\f[p]{v_1} \gets \f[p]{v_1} - \gamma$ % \Comment{Price update}
            \If{$\abs{\lrc{e \in F \colon v_1 \in e}} = \f[b]{v_1}$} \Comment{$v_1$ is saturated}
                \State Choose $y$ randomly from the set of bidders $v_1$ is matched to
                \State $F \gets F \sm \lrc{\lrp{y, v_1}} \cup \lrc{\lrp{u, v_1}}$ \Comment{Remove previous edge and insert new one}
                \State $U \gets U \cup \lrc{y}$   
            \Else 
                \State $F \gets F \cup \lrc{\lrp{u, v_1}}$
            \EndIf
            \If{$\abs{\lrc{e \in F \colon u \in e}} = \f[b]{u}$} \Comment{$u$ is saturated}
                \State $U \gets U \sm \lrc{u}$  
            \EndIf 
        \EndProcedure
    \end{algorithmic}
\end{algorithm}

\begin{observation}
    In each iteration of \cref{alg:auction}, only one object has a price update. The price of this object decreases by at least $\varepsilon$. 
    \label{obs:price-update}
\end{observation}

The choice of the bidding value $\gamma = \f[w^\prime]{u, v_2} - \f[w^\prime]{u, v_1} + \varepsilon$ for an unsaturated bidder $u$
is made to maintain a simple $b$-matching that is $\varepsilon$-happy with respect to the price function $p$ in each iteration. 

\begin{lemma}
    %At the start of any iteration of \cref{alg:auction}, $F$ is $\varepsilon$-happy with respect to the price function at that iteration. 
    If at the start of an iteration of \cref{alg:auction} $F$ is $\varepsilon$-happy with respect to the price function $p$, the updated 
    matching is still $\varepsilon$-happy with respect to the updated price function at the end of the iteration. 
    \label{lem:F-eps-happy}
\end{lemma}
\begin{proof}
    Suppose that in this iteration, bidder $i$ finds objects $j$ and $\overline{j}$ that have the lowest and second-lowest augmented cost, respectively, 
    and makes a bid on object $j$. 
    Let $p$ and $p^\prime$ be the price function before and after the bid, respectively. Since the bid 
    $\gamma = \f[w^\prime]{i, \overline{j}} - \f[w^\prime]{i, j} + \varepsilon$, we get that 
    \begin{align*}
        \f[p^\prime]{j} = \f[p]{j} - \lrp{\f[w^\prime]{i, \overline{j}} - \f[w^\prime]{i, j} + \varepsilon}
        %&= \f[p]{j} + \f[w^\prime]{i, j} - \f[w^\prime]{i, \overline{j}} - \varepsilon\\
        = \f[w]{i, j} - \f[w^\prime]{i, \overline{j}} - \varepsilon
    \end{align*}
    and hence 
    \begin{align*}
        \f[w]{i, j} - \f[p^\prime]{j} &= \f[w^\prime]{i, \overline{j}} + \varepsilon
        = \min_{k \in N(i) \sm \lrc{j}} \lrp{\f[w]{i, k} - \f[p]{k}} + \varepsilon.
    \end{align*}
    By \cref{obs:price-update}, we find that for all $r \in R$, $\f[p^\prime]{r} \leq \f[p]{r}$ with equality for all objects besides $j$. 
    Hence, 
    \begin{align*}
        \f[w]{i, j} - \f[p^\prime]{j} = \min_{k \in N(i) \sm \lrc{j}} \lrp{\f[w]{i, k} - \f[p]{k}} + \varepsilon \leq \min_{k \in N(i)} \lrp{\f[w]{i, k} + \f[p^\prime]{k}} + \varepsilon
    \end{align*}
    so the edge $\lrp{i, j}$ is $\varepsilon$-happy with respect to $p^\prime$.  
    Additionally, for each edge $\lrp{a, b}$ in $F$ where $b \neq j$, we find that 
    \begin{equation*}
        \f[w]{a, b} - \f[p^\prime]{b} = \f[w]{a, b} - \f[p]{b} \leq \min_{c \in N(a)} \lrp{\f[w]{a, c} - \f[p]{c}} + \varepsilon \leq \min_{c \in N(a)} \lrp{\f[w]{a, c} - \f[p^\prime]{c}} + \varepsilon.
    \end{equation*}
    Thus, $F$ is also $\varepsilon$-happy with respect to $p^\prime$.
    Therefore, the updated matching $F^\prime = F \cup \lrc{\lrp{i, j}}$ (plus any additional removals if $j$ is already saturated) is $\varepsilon$-happy with respect to $p^\prime$. 
\end{proof}